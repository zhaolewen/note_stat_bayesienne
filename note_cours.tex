\documentclass{article}
\linespread{1.5}

\usepackage{amsmath}
\usepackage{amsfonts}
\usepackage{mathrsfs}

\title{Statistique Bay\'esienne}
\setcounter{section}{-1}

\begin{document}
\maketitle

\tableofcontents
\pagebreak

\section{Introduction}
\section{Inference Bay\'esienne}
\subsection{Inference statistique et th\'eorie de la d\'ecision}
def: Modele statistique

\begin{equation}
u = (Y,F,P)
\end{equation}
	
Si P est (?) de loi a (?), les (?) 

$P_\theta$
pt de depend de l'inference statistique:
ou cherche a "(?)" la valeur d'une (?)

g(theta)\ in Z est (?) le concept d'(?)

def: un estimateur ou regle de decision, thtre facteurs (?)
delta: Y->Z

On veut construire $delta$ de sorte que ayant observe "$Y=y$", $\delta (y)$ sait une "(?)" approximative de $g(\theta)$

def: On appelle fonction de perte, une fonction 

\begin{equation}
	L: P\times -> \mathbb{R}_+
\end{equation}
ou
\begin{equation}
	L:\Theta\times Z -> \mathbb{R}_+
\end{equation}

dans le cas d'une famille paramétrique.

et telle que

i) $\forall \theta \in \Theta$
$L(\theta,g(\theta)) = 0$

ii)si l'absence $Y=y$ et que l'on (?) le regle de decision $\delta$, alors la quantite $L(\theta,\delta(y))$ represente le coeur associe a la decision
$S(y)$ pour la loi $P_\theta \in \mathscr{P}$

Archetype de fonction de perte: perte quadratique
\begin{equation}
	L(\theta, \delta(y)) = (g(\theta) - s(y))^2
\end{equation}

- autre fonctions de pertes:

value absolue $C^1$, pertes 0-1 (tests d'hypothese)

(?) pertes joules ou l'entropie

(?) en euros

def: la performance de la regle de decision $\delta$ est quanti-free a (?), definie comme la perte moyenne

\begin{equation}
	R(\theta,S) = E_{y\in \mathscr{P}_\theta} \{L(\theta, \delta(y))\} = \int_Y L(\theta,\delta(y))d P_\theta(y)=\int_y L(\theta,\delta(y))P_\theta(y)dy
\end{equation}

\subsection{Methode de construction (?)}
\subsubsection{(?) d'une regle optimal dans une sous-classe}

objectif: construire un $\delta^*$ dans une classe de regles de decision $\tau$ telle que $\forall \delta \in \tau$

\begin{equation}
	R(\theta,\delta^*)<=R(\theta,\delta)
\end{equation}
pour tout $\theta\in\Theta$

Cas particulier important

-> recherche d'estimateurs sans biais de variance minimale

-> de tels elements optimale (?), dans le cadre des (?) (voir cours 1A)

\subsubsection{Optimisation d'un critere}

-> recherche une estimateur avec minimiseur ou maximiseur d'un critere


\section{Algorithme de (?)}

MM a 

en utilisant une densite $q(x|x_1)$ appele 
-densite instrumentale
-densite de condition
-densite proproditionlelle

-> doit etre facile a simuler
->peut etre tres differente de p (la cible)

(graph ici)

exemple gaussien centre sur $x_i$

MH:

al'iterition i $(x_1, .. ,x_n)$ a deja ete 
1) generer y_i ~ q(. | x_i)
2) calcule le rapport d'accpetation 
\begin{equation}

a=\frac{P(y_i)}{P(x_i)}\frac{q(x_i|y_i)}{q(y_i|x_i)}

\end{equation}

3) si a >= 1

prendre x_{i+1} = y_i

sinon

\{
x_i+1=y_i avec (?)
x_i+1=x_i


remarque
1) (?)
2) si q est symmetrique
$q(x|y) = q(y|x)$
alor $a=\frac{p(y)}{p(x)}$
->tendence a se diriger dans les zones de densite p eleve

(graph ici)

l'algorithme de MR est un procede d'exploration de l'espace e
-> l'analyse de cet alfgorithem repose sur la theorie de chaine de Markov

-> culture
->analyse MH


\subsection{Element de theorie des chaines de Markov}
\subsubsection{}

def Soit(E,\eplsion) un espace mesurable. Une (?) sur E

une famille de probabilite

\begin{equation}

\{K(x,.)_{x\in E}
\end{equation}

indexe par les points de E.

Pout tout ensemble mesurable $\forall \lambda \in \E$

l'application x-> $K(x,A)$ est mesurable.

$\forall x \in E, \forall A\in \Sigma$

$K(\sigma, A)$ represente une probabilite de x (?) dans A a partir de x

Etant donne un noyeau K et une probabilite $\mu$ sur $E$, on definit une nouvelle probabilite $\mu K$ par $\forall A \in E$
\begin{equation}

\mu K(A) = \int_E \mu (dx) K(x,A)
\end{equation}

NB: 
\begin{equation}
E(x) = \int x dP_x(x) = \int P_x(dx)
\end{equation}

En particulier, si $mu$ admet une densite $g$

\begin{equation}
\mu K(A) = \int_E K(x,A) g(x) dx
\end{equation}

Composition de noyeau

def: soit $K_1$ et $K_2$ deux noyeaux, on definit le  noyeau

$K_1 K_@$ pour $\forall (x,A)\in (E,\Epsilon)$

\begin{equation}
(K_1 K_2)(x,A) = ((K_1,.)K_2)(A)=\int_E K_1(x,dy)K_2(y,A)
\end{equation}

Si $K1$ a une densite k_1

\begin{equation}
=\int_E k_1(x,y)K_2(y,A)dy
\end{equation}

- $K_1K_2(x,A)$ 
- en faisant une 1ere $K_1$ et (?)

OIn definit par recurrence les noyeaux itere

$\forall x > 1, \forall (x,A) \in E \times \Sigma$


\begin{equation}
K^n(x,A) =\int_E K^{n-1}(x,dy)K(y,A)
\end{equation}

avec $K^0(X,A)= \mathbb{1}_A(x)=\delta_x(A)$

\subsubsection{Chaine de Markov}

def: Soit $(\Omega, A,P)$ une (?)
Une suite $(X_n)_n\in \mathbb{N}$ (?)

est une chaine de Markov s'il existe une suite de noyaux $(K_n)_{n\in \mathbb{N}}*$ (?)

$\forall n,A_0,A_1, ... ,A_n \in \Epsilon$

$P(x_0 \in A_0, ... x_n \in A_n)=\int_{x_0\in A_0} ... \int_{x_n \in }A_n \mu(d x_0)K_1(x_0,dx1) .. K_n(X_{n-1},dx_n)$
ou $\mu$ est la loi de $X_0$ applique la (?) de la chaine

-> On a deduit que la loi $P_{x_{n+1}}$ est la loi de $P_{x_n}K_{n+1}$

Si E est discret, (E est fini ou denombrable)
On utilise la definition suivante

def: $x_n$ est une chaine de Markov sur une chaine discret si 

\begin{equation}
P(X_{x=x_0},...,X_{x=x_n})=P(X=x_0)K_1(x_0, x_1),..., K_n(x_{n-1},x_n)
\end{equation}

$\forall x_0,...,x_n \in E$
on a (?)

\begin{equation}
P(X_n=x_n|X_1=x_1,...,X_{n-1}=x_{n-1})=K(x_{n-1,x_n})
\end{equation}

la loi conditionnelle de $X_n$ connaissance l'histoire du processus ne depende que de $X_{n-1}$

def:Lorsque $K_n$ ne depend pas de $n$ -> La chaine ($x_n$) est dit homogene(synome de stationnaire)

\subsection{Prpbabilite invariante}

->Regarder la notion de convergence de la chaine de Markov

def: Si $(X_n)$ est une chaine de Markov homog\`ene, alors la limite $\Pi \in M_1^+(E)$ du systeme dynamique definit par 
$P_{x_n}=P_{x_{n-1}}K$  si elle (?) est un poiunt fixe de la transformation lineaire

\begin{equation}
\Pi \in (?) -> \Pi K
\end{equation}

c'est a dire, $\forall A \in \epsilon$

\begin{equation}
\Pi(A)=\int \P_i(dx)K(x,A)
\end{equation}

une telle probatilite $\Pi$ est appele (?) invariante (?) de $(X_n)$

Objectif:Etudier le comportement asymptotique de $P_{x_n}$

- Condition suffisante qui garantit que une loi $\Pi$ est invariante.

def: Une chaine de Markov de noyau $K$ est $\Pi$-reversible si $\forall A,B \in \Epsilon$

\begin{equation}
\int_B K(x,A)\Pi(dx) = \int_A K(x,B)\Pi(dx)
\end{equation}
(Probabilite de B-> A = Probabilite de A->B)

Remarque Si $E$ est (?)
$\forall x,y \in E$

\begin{equation}
\Pi(x)K(x,y)=\Pi(y)K(y,x)
\end{equation}
On dit ainsi que $\Pi$ est reversible.

Proposition: Si $\Pi$ est reversible, alors cette $\Pi$ est invariante.

demonstration:

Si $B=E, \foprall A \in \Sigma$

\begin{equation}
\int_A K(x,E)\Pi(dx)=\Pi(A)=\int_E K(x,A)\Pi(dx)
\end{equation}

Les noyeau des algorithmes MCMC et en particulier celui de l'algorithme de MH sont quasiment toujours construits de maniere a verifier la reversibilite pour la densite cible $P$.

Etape suivante: etudier la convergence d'une chaine de Markov vers une loi invariante.

\subsection{Propriete de convergence}

1ere idee: pour qu'une chaine converge, il faut assurer que l'on peut atteindre "n'importe" etape partant d'une condition initiale quelquconque.

def: le temps d'entree (liting time) $\tau_A$ d'une chaine $(X_n),n\in \mathbb{n}$ dans $A\in \Sigma$ est defini par

$\tau_A= of n>=1, X_n\inA$

def: une chaine de Markov est dite phi-inreductible pour une probilite $\phi$ si $\forall x \in E, \forall A\in \Sigma$, on a

\begin{equation}
\phi(A)>0 => P(P(\tau_A<\inf|X_0=x)) >0
\end{equation}

signifie que tout ensemble de $\phi_n$ probabilite non nulle peut etre atteindre a partie de n'importe quelle valeur initiale de $X_0$.     

Theoreme: Soit ($X_n$) une chaine de Markov $\Pi$-inreductible et $\Pi$-invariante (?)


\begin{equation}
\int_E |\tau(x)|\Pi(dx)<-\inf
\end{equation}

alors pour presque tout points initiale $x_0\in E$


\begin{equation}
\hat{\phi_n}=\frac{1}{n}\sum_{i=1}^{n}\phi(X_n) converge p.s. \Phi=\int \phi(x)\Pi(dx)
\end{equation}

-> Loi forte de grande nombre pour une chaine de Markov.

On peut obtenir des resultats plus forts en "eliminant" les comportement deterministes d'une chaine de Markov.

Def: aperidocite

Une chaine de Markov $(X_n)$ $\phi$-irreductible est aperidodique s'il n'existe pas de partition de E de la forme $(E_1,...,E_n,N)$

avec $Nlog\phi(N)=0$ et $E_1,...,E_n$tel que

\begin{equation}
\forall n>0 P(X_{n+1}\in E_{n+1}|X_n\in E_n) = 1
\end{equation}

-> pas des contraintes deterministe.

Theoreme: Soit une chaine de Markov de noyeau de transition $K$, $\Pi$-irreductible, $\Pi$-irreversible, alors pour $\Pi$ tout points initiale $x_0 \in E$

\begin{equation}
lim_{n->+\inf} ||K^n (x_0, \dot)\Pi||_{VT}=0
\end{equation}

VT: Variation Totale

\section{Algorithme MCMC (MH)}
\subsection{Convergence de MH}
Rappel:
Def: Etant donn\`e une loi instrumentale $q$, 
i)generer un $i_k~q(y|x_k)$
ii) poser

\begin{equation}

\end{equation} 
avec proba $a=\rou(x_k,y_k)$, $x_k$ avec proba 1-a

si 
\begin{equation}
\rou(x,y) = min\{1,\frac{p(y)}{p(x)}\frac{p(x|y)}{p(y|x)}\}
\end{equation}

proposition: 
l'algorithme de MH (?) construit un noyeau de
\begin{equation}
K_{MH}(x,dz=q(z|x))\rou(x,z)dz+(1-x(x))\delta_x(dy)
\end{equation}
avec
\begin{equation}
\delta(x)=\int_E \rou(x,y)q(y|x)dy
\end{equation}

















\end{document}