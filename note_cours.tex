\documentclass{article}
\linespread{1.5}

\usepackage{amsmath}
\usepackage{amsfonts}
\usepackage{mathrsfs}

\title{Statistique Bay\'esienne}
\setcounter{section}{-1}

\begin{document}
\maketitle

\tableofcontents
\pagebreak

\section{Introduction}
\section{Inference Bay\'esienne}
\subsection{Inference statistique et th\'eorie de la d\'ecision}
def: Modele statistique

\begin{equation}
u = (Y,F,P)
\end{equation}
	
Si P est (?) de loi a (?), les (?) 

$P_\theta$
pt de depend de l'inference statistique:
ou cherche a "(?)" la valeur d'une (?)

g(theta)\ in Z est (?) le concept d'(?)

def: un estimateur ou regle de decision, thtre facteurs (?)
delta: Y->Z

On veut construire $delta$ de sorte que ayant observe "$Y=y$", $\delta (y)$ sait une "(?)" approximative de $g(\theta)$

def: On appelle fonction de perte, une fonction 

\begin{equation}
	L: P\times -> \mathbb{R}_+
\end{equation}
ou
\begin{equation}
	L:\Theta\times Z -> \mathbb{R}_+
\end{equation}

dans le cas d'une famille paramétrique.

et telle que

i) $\forall \theta \in \Theta$
$L(\theta,g(\theta)) = 0$

ii)si l'absence $Y=y$ et que l'on (?) le regle de decision $\delta$, alors la quantite $L(\theta,\delta(y))$ represente le coeur associe a la decision
$S(y)$ pour la loi $P_\theta \in \mathscr{P}$

Archetype de fonction de perte: perte quadratique
\begin{equation}
	L(\theta, \delta(y)) = (g(\theta) - s(y))^2
\end{equation}

- autre fonctions de pertes:

value absolue $C^1$, pertes 0-1 (tests d'hypothese)

(?) pertes joules ou l'entropie

(?) en euros

def: la performance de la regle de decision $\delta$ est quanti-free a (?), definie comme la perte moyenne

\begin{equation}
	R(\theta,S) = E_{y\in \mathscr{P}_\theta} \{L(\theta, \delta(y))\} = \int_Y L(\theta,\delta(y))d P_\theta(y)=\int_y L(\theta,\delta(y))P_\theta(y)dy
\end{equation}

\subsection{Methode de construction (?)}
\subsubsection{(?) d'une regle optimal dans une sous-classe}

objectif: construire un $\delta^*$ dans une classe de regles de decision $\tau$ telle que $\forall \delta \in \tau$

\begin{equation}
	R(\theta,\delta^*)<=R(\theta,\delta)
\end{equation}
pour tout $\theta\in\Theta$

Cas particulier important

-> recherche d'estimateurs sans biais de variance minimale

-> de tels elements optimale (?), dans le cadre des (?) (voir cours 1A)

\subsubsection{Optimisation d'un critere}

-> recherche une estimateur avec minimiseur ou maximiseur d'un critere

\end{document}